\documentclass[letterpaper,11pt,reqno]{amsart}

% latex commands and macros

% common sets
\newcommand{\R}{\mathbb{R}} % the set of real numbers
\newcommand{\C}{\mathbb{C}} % the set of complex numbers
\newcommand{\Z}{\mathbb{Z}} % the set of integers
\newcommand{\N}{\mathbb{N}} % the set of natural numbers
\newcommand{\Q}{\mathbb{Q}} % the set of rational numbers
\newcommand{\F}{\mathbb{F}} % usually stands for field
\newcommand{\D}{\mathbb{D}} % the unit disc in the complex plane
\newcommand{\B}{\mathbb{B}} % the unit ball
\newcommand{\es}{\mathbb{S}} % sphere
\newcommand{\T}{\mathbb{T}} % torus
\newcommand{\RP}{\mathbb{R}P} % projective plane
\newcommand{\CP}{\mathbb{C}P} % complex projective plane
\DeclareMathOperator \Gr {Gr} % grassmannian
\newcommand{\ham}{\mathbb{H}} % quaternions

% common operations and connectors
\newcommand{\inner}[2]{\left<{#1},\,{#2}\right>} % inner product
\newcommand{\coset}[1]{\left<{#1}\right>} % groups or ideal generated by some set
\newcommand{\norm}[1]{\left\|{#1}\right\|} % norm of a vector
\newcommand{\abs}[1]{\left|{#1}\right|} % absolute value
\newcommand{\sets}[1]{\left\{{#1}\right\}} % a simple set with no conditions
\newcommand{\set}[2]{\sets{{#1}:\,{#2}}} % a set with conditions
\newcommand{\prn}[1]{\left({#1}\right)} % parentheses 
\newcommand{\brak}[2]{\left[{#1},{#2}\right]} % bracket product
\newcommand{\pbrak}[2]{\left\{{#1},{#2}\right\}} % Poisson bracket product

\DeclareMathOperator \csum {\#} % connected sum

\DeclareMathOperator \pp {\,a.e.} % almost everywhere
\DeclareMathOperator \sgn {sgn} % signum function
\DeclareMathOperator \spn {span} % span
\DeclareMathOperator \cspn {\overline{span}} % closed span
\DeclareMathOperator \img {Im} % imaginary part of a complex number
\DeclareMathOperator \rea {Re} % real part of a complex number
\DeclareMathOperator \ind {ind} % index of a function
\DeclareMathOperator \diam {diam} % diameter of a set or a graph
\DeclareMathOperator \res {res} % residue of a complex function
\DeclareMathOperator \dist {dist} % distance function
\DeclareMathOperator \wind {wind} % winding number
\DeclareMathOperator \spectrum {\sigma} % spectrum of an operator
\DeclareMathOperator \spectrumEssential {\sigma_{\text{ess}}} % essential spectrum of an operator

\DeclareMathOperator \smb {smb} % symbol $C^*$-homomorphism

\newcommand{\acton}[2]{\phantom{|}^{{#1}}{{#2}}} % hack to prepend a superscript

\DeclareMathOperator \Syl {Syl} % Sylow subgroup
\DeclareMathOperator \im {im} % image of a function
\DeclareMathOperator \Tor {Tor} % torsion
\DeclareMathOperator \Aut {Aut} % automorphisms
\DeclareMathOperator \End {End} % endomorphisms
\DeclareMathOperator \Mod {Mod} % modules
\DeclareMathOperator \Hol {Hol} % holomorphic functions
\DeclareMathOperator \Sym {Sym} % symmetric group
\DeclareMathOperator \Alt {Alt} % alternating groups
\DeclareMathOperator \Mat {Mat} % matrices
\DeclareMathOperator \Spec {Spec} % spectrum of a ring
\DeclareMathOperator \Max {Max} % maximal spectrum of a ring
\DeclareMathOperator \coker {coker} % cokernel
\DeclareMathOperator \GL {GL} % general linear group
\DeclareMathOperator \SL {SL} % special linear group
\DeclareMathOperator \codim {codim} % codimension of a vector subspace
\DeclareMathOperator \rnk {rnk} % rank of a linear transform
\DeclareMathOperator \Id {Id} % identity function
\DeclareMathOperator \Tr {Tr} % trace of a matrix
\DeclareMathOperator \diag {diag} % diagonal matrix
\DeclareMathOperator \supp {supp} % support of a function

% category theory
\DeclareMathOperator \Ob {Ob} % objects
\DeclareMathOperator \Mor {Mor} % morphism
\DeclareMathOperator \Hom {Hom} % homomorphisms
\DeclareMathOperator \Lin {\mathbf{Lin}} % linear
\DeclareMathOperator \Set {\mathbf{Set}} % sets
\DeclareMathOperator \Vect {\mathbf{Vect}} % vector spaces

% vector calculus
\DeclareMathOperator \diver {div} % divergence
\DeclareMathOperator \grad {grad} % gradient
\DeclareMathOperator \curl {curl} % curl

\DeclareMathOperator \Lie {Lie} % Lie groups
\newcommand{\lie}[1]{\mathfrak{{#1}}} % lie algebra

% operator analysis
\newcommand \slim {\text{s-}\lim} % strong limit
\DeclareMathOperator \Fl{\mathbb{F}l} % spaces of frequency weighted norms
\DeclareMathOperator \Smb{Smb} % symbol operator
\DeclareMathOperator \Alg{Alg} % algebra generated by a set

% statistics
\DeclareMathOperator \Exp{\mathbb{E}} % variance
\DeclareMathOperator \Var{Var} % variance

% miscellaneous

% frequently used abbreviations
\newcommand{\ve}{\varepsilon} % epsilon clearly distinguishable from inclusion
\newcommand{\vf}{\varphi} % more popular way of writing phi
\newcommand{\vn}{\varnothing} % the empty set

% special notation for dissertation
\newcommand{\hprod}{\circ} % Hadamard or entrywise multiplication of matrices
\newcommand{\hs}{{\mathcal{C}_2}} % Hilbert-Schmidt class
\newcommand{\poi}{{\mathcal{P}_1}} % Poincare class
\newcommand{\MellinTransform}{\mathcal{M}} % Mellin Transform
\newcommand{\MellinConvolution}{*_{\MellinTransform}} % Mellin convolution

% environments and fonts for cifar-ten project
\newcommand{\code}[1]{\texttt{{#1}}}

% styles of theorems, definitions and remarks
\theoremstyle{plain}
\newtheorem{thm}{Theorem}%[section]
\newtheorem{lemma}[thm]{Lemma}
\newtheorem{prop}[thm]{Proposition}
\newtheorem{cor}[thm]{Corollary}
\newtheorem{axiom}[thm]{Axiom}
\newtheorem{exercise}[thm]{Exercise}
\newtheorem{claim}[thm]{Claim}
\newtheorem{fact}[thm]{Fact}

\theoremstyle{definition}
\newtheorem{defn}[thm]{Definition}

\theoremstyle{remark}
\newtheorem{remark}[thm]{Remark}
\newtheorem{exmpl}[thm]{Example}
\newtheorem{problem}[thm]{Problem}
\newtheorem{prob}[thm]{Problem}
\newtheorem{fle}[thm]{File}

% page format
\oddsidemargin=0in
\evensidemargin=0in
\textwidth=6in
\topmargin=-0.5in
\textheight=9.5in
\parindent=.375in

% packages
\usepackage{enumerate}
\usepackage{amssymb}
%\usepackage[all,cmtip]{xy}
\usepackage[mathscr]{eucal}
\usepackage{graphicx,color}

\newcommand{\titleChaos}[1]{\title[HW {#1}]{Introduction to Chaos, Homework {#1}}}

%\begin{figure}[ht]
%       \includegraphics[height=1in,keepaspectratio]{03NegativeHelix.jpg}
%       \caption{Negative writhe crossings} \label{fig:negativeWritheCrossings}
%\end{figure}

\usepackage{hyperref}

\begin{document}

\author{Authors TBD}
\title{Tuning DBN hyper-parameters using a genetic algorithm}
\date{\today}
\maketitle

\section{Introduction}

Deep belief networks (DBNs) are neural networks which are first pre-trained in a greedy layer wise manner using Boltzmann learning and then fine tuned using back-propagation. DBN’s implement state of the art image and speech recognition. While the results can be impressive, 

The significant increase in performance of these DBNs over traditional neural networks comes at a cost of an increase of hyper-parameters which must be tuned to ensure even good performance. The process of tuning these hyper-parameters can be tedious, time consuming and expensive. Even worse, whether a set hyper-parameters works well also may depend on what hardware is used to train the classifier.


\section{Laundry list of hyper-parameters}

Random Seed 

Classifier learn rate 

Classifier l2-penalty 

Classifier l1-penalty 

Number of layers (n. layers) 

Number of hidden units (n. hidden-units) 

(Mini) Batch size 

Weight initialization method 

CD epochs

CD learn rate 

Convolution networks: (additional constraint on the weights)? 

Cost Function

Layer by layer fitness function evaluation

Activation function

Pre processing strategy: PCA, Fourier Transform, et al.  

Defining the hyper-parameters: 

As identified Supra. We will define each hyper parameter to clarify the process in which it’s assigned. 

Random Seed –  a selection of random numbers or vectors that is used to generate pseudo-random numbers.

Classifier Learn Rate – how quickly a gradient moves when updating any given weight. 

Classifier L2-penalty –  a term that is added to the cost function to ensure regularization. 

Classifier L1-penalty –  a second term that is added to the cost function to ensure regularization. 

Number of layers – Roughly corresponds to the level of hierarchical abstraction for the overall network. 

Number of hidden units – Roughly corresponds  to the amount of features that may be processed in each layer.

(Mini) Batch size – The number of samples selected from the trading set to evaluate the gradient decent at any given time. 

Weight initialization method procedure (WIMP) – Strategy for introducing the initial weights. 

CD Epochs – The maximum number of iteration during Boltzmann optimization.

CD Learn Rate – Learning right for the Boltzmann optimization. 

Convolution networks  - Whether to use or not to use a convolutional network (using one introduces a whole new set of hyper-parameters). 

Cost Function – Which type of cost function to use, such as: logistic, linear, etc. 

Layer by layer fitness function evaluation – Whether to use it .  

Activation Function – Such as: sigmoid, arc tan, etc.

Pre Processing Strategy – whether to use raw data or to pre process it using: PCA, Fourier Transform, et al. 

\section{Genetic algorithm and multi-objective optimization}

\section{Methods}

\section{Results}

\section{Future work}

\section{Reading}
\nocite{*}
\bibliographystyle{amsplain}
\bibliography{cifarten}

\end{document}
